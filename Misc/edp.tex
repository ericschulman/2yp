\documentclass{article}
\usepackage[utf8]{inputenc}
\usepackage{amsmath}
\usepackage{graphicx}

\title{Graphing EDP}
\author{Eric Schulman}
\date{July 2017}

\DeclareMathOperator{\E}{\mathrm{E}}
\DeclareMathOperator{\Var}{\mathrm{Var}}

\begin{document}

\maketitle

\section{Functional Form of $p(t)$}

The following section defines $p(t)$, the price function for the manufacturer price. 

The function $p(t)$ is a piecewise function that takes on a low price (promotion) and a high price (everyday price). Because of the cyclical nature of promotions, prices are periodic. The duration of each sales cycle is $\lambda$ units of time and the duration of each promotion is $\omega$ units of time. After the promotion is over, the price goes back to its everday level for the rest of the period.


$$p(t) =  \begin{cases} p_{promo} & t \bmod  \lambda \leq \omega \\ p_{edp} &  t \bmod  \lambda > \omega \\ \end{cases} $$

For convenience, we can normalize the prices and normalize the length of the period. 

The normalized prices are
$$\delta = \dfrac{p_{promo}}{p_{edp}}$$ 
Where $0 \leq \delta \leq 1$

Additionally, the normalized periods are
$$\theta = \dfrac{\omega}{\lambda}$$ 
Where $0 \leq \theta \leq 1$

So, $p(t)$ normalized in terms of $\delta$ and $\theta$ is 
$$p(t) =  \begin{cases} \delta & t \bmod  1 \leq \theta \\ 1 &  t \bmod  1 > \theta \\ \end{cases} $$


\begin{figure}
\centering
\includegraphics[scale = .5]{figure_1}
\caption{This figure shows a graph of the wave. In the figure, $p_{edp} = 1$ and $p_{promo} = .5$. Additionally, in this figure $\omega = .5$ and $\lambda = 1$}
\label{figure_1}
\end{figure}


\section{Calculating the Average}

\subsection{Average Price Over One Cycle}

We can calculate the expected price of $p(t)$. We start by calculating the average price $p(t)$ over one period i.e. $t\in[0,1]$

$$\E[p(t)] = \int_{0}^{1} p(t) dt $$
$$\E[p(t)] = \int_{0}^{\theta} \delta dt + \int_{\theta}^{1} dt$$
$$\E[p(t)] = \delta \theta + 1 - \theta $$
$$\E[p(t)] = 1 - (1-\delta)\theta $$

\subsection{Average  Price Over $n$ Cycles}

We can extend our average price to $n$ cylces i.e. $t\in[0,n]$ and $n$ is an integer. 

$$E(p(t)) = \frac{1}{n} \int_{0}^{n} p(t) dt $$
However, because of the cyclical nature of $p(t)$ we can evaluate the integral by repeating the first period $n$ times.
$$E(p(t)) = \frac{1}{n} n \int_{0}^{1} p(t) dt $$
$$E(p(t)) = \int_{0}^{1} p(t) dt $$
$$\E[p(t)] = 1 - (1-\delta)\theta $$

\subsection{Average  Price with an Offset}

If the cycles are offset by $\tau$ we compute the average as follows.

$$E(p(t)) = \frac{1}{n} \int_{0}^{n+\tau} p(t) dt $$
$$E(p(t)) = \frac{1}{n+\tau} (n \int_{0}^{1} p(t) dt +  \int_{0}^{\tau} p(t) dt) $$
$$E(p(t)) = \frac{1}{n+\tau} (n (1 - (1-\delta)\theta) + \int_{0}^{\tau} p(t) dt)$$

In the case where $\tau \leq \theta$ then

$$E(p(t)) = \frac{1}{n+\tau} (n (1 - (1-\delta)\theta) + \int_{0}^{\tau} \delta dt)$$
$$E(p(t)) = \frac{1}{n+\tau} (n (1 - (1-\delta)\theta) + \delta\tau)$$

In the case where $\tau > \theta$

$$E(p(t)) = \frac{1}{n+\tau} ( n(1 - (1-\delta)\theta) + \int_{0}^{\theta} \delta dt + \int_{\theta}^{\tau}) dt $$
$$E(p(t)) = \frac{1}{n+\tau} (n (1 - (1-\delta)\theta) + \delta \theta + \tau - \theta) $$
$$E(p(t)) = \frac{1}{n+\tau} (n (1 - (1-\delta)\theta) + \tau - (1-\delta)\theta ) $$

\subsection{Average  Price}

In both cases we can see that as $n$ get abirtrarily large, $\E[p(t)]$ approaches $1 - (1-\delta)\theta$.

So, $$\E[p(t)] = 1 - (1-\delta)\theta $$

\section{Calculating the Variance}

We can calculate the variance of the promotions in each period as well. Since the promotions repeat themselves, each period will have the same varaince.

$$\Var[p(t)] = \E[p(t)^2] - (\E[p(t)])^2$$

First compute $\E[p(t)^2]$

$$\E[p(t)^2] = \int_{0}^{1} p(t)^2 dt $$
$$\E[p(t)^2] = \int_{0}^{\theta} \delta^2 dt + \int_{\theta}^{1} dt$$
$$\E[p(t)^2] = \delta^2 \theta + 1 - \theta $$
$$\E[p(t)^2] = 1 - (1-\delta^2)\theta $$

Then compute $\E[p(t)]^2$

$$\E[p(t)]^2 = (1 - (1-\delta)\theta)^2 $$

Combining these results,

$$\Var[p(t)] = 1 - (1-\delta^2)\theta -  ( 1 - (1-\delta)\theta )^2$$

Simplifying we have,

$$\Var[p(t)] = (1- \delta)^2 (\theta - \theta^2)$$


%$$\Var[p(t)] = E[ (p(t) - E[p(t)])^2 ] $$
%$$\Var[p(t)] = \int_{0}^{1} ( p(t) - E[p(t)] )^2 $$
%$$\Var[p(t)] = \int_{0}^{\theta} ( \delta - E[p(t)] )^2  dt + \int_{\theta}^{1} ( 1 - E[p(t)] )^2 dt $$
%$$\Var[p(t)] = ( \delta - E[p(t)] )^2 \theta +  ( 1 - E[p(t)] )^2 (1-\theta) $$
%(d - 1)^2 g (1 - g)
%(d - 1)^2 (g - g^2)



\end{document}