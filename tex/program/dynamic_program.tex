\documentclass{article}

%math stuff
\usepackage{amsmath}
\usepackage{enumitem}
\usepackage{mathtools}

%bibliography/appendix
\usepackage{cite}
\usepackage[toc,page]{appendix}

%figures
\usepackage{graphicx}
\usepackage{booktabs}

%General Formating
\renewcommand*\familydefault{\sfdefault}
\usepackage[letterpaper, portrait, margin=1.5in]{geometry}
\usepackage{fancyhdr}
\pagestyle{fancy}

%Header
\lhead{Schulman}
\rhead{Page \thepage}

\title{Consumer Purchases with Promotions as a Dynamic Program}
\author{Eric Schulman}
\date{\today}

\begin{document}

\maketitle

\section{Introduction}
Gonul and Srinivasan seek to predict consumer purchases during by estimating the associated costs of buying the product, and the costs of waiting for a promotion and 'stocking out' in the interim. These costs are calculated using the premise that consumers anticipate future promotions and the savings associated with them. As a result, each time the consumer calculates the costs of buying or 'stocking out', they dynamically consider future costs. In their model, the consumer predicts these future costs based on a recursive cost function $V(B_t)$. In the function $B_t$ represents a binary decisions on whether or not the consumer bought the product at tme $T$.

The purpose of this document is to explain how to modify $V(t)$ function to use volume data. It includes a worked out example of one good with two periods. This document also explains at a high level how to estimate the parameters to $V(t)$ using maximum likelihood estimation.


\section{Update Rule}

Because the choices are not binary, you need to keep track of the amount of goods the consumer has purchased. At time $t$, the consumer has inventory $I_t$ which is the sum of the purchases $X_t$ at time $t$, the previous inventory $I_{t-1}$ and an amount $K$ that is consumed in each period. This amount is assumed ot be constant. The update rule can be modified, to be more naunced. 

$$I_t = X_t + I_{t-1} - K$$

The consumer's value function is given by a cost for over or under consuming. This is $C(I_t)$, it is essentially an inventory cost. Then 



$$\text{Minimize } V(I_{t-1}, P_t(t)) = C(I_t) + P_t(t) X_t + V(I_t, P_{t+1}(t) ) $$

\section{Worked Out Example}

\[ 
P(t) =
\begin{cases}
p_1 & t = 1 \\ 
p_2 & t = 2
\end{cases} 
\]

$$I_0 = 0$$

$$ C(I_t) = \alpha (I_t)^2$$

$$\text{Minimize }  \alpha (I_1)^2 + p_1 X_1 +  V(I_1,P_2(2)) $$


$$\text{Minimize }  \alpha (I_1)^2 + p_1 X_1 +  \alpha (I_2)^2 + p_2 X_2 $$

Plugging in the update rule

$$\text{Minimize }  \alpha (X_1 + I_0 - K )^2 + p_1 X_1 +  \alpha (X_2 + (X_1 + I_0 - K)  - K )^2 + p_2 X_2 $$

We can work out this example using first order conditions to see

$$X_1 = \frac {2\alpha K - p_1 + p_2}{2 \alpha } $$

$$X_2 = \frac {p_1 - p_2}{2 \alpha } $$


\section{Maximum Likelihood Estimation}

$$\epsilon_t = X_t^{obs}- \hat{X_t}$$

$$ \text{Maximize }_{\alpha, K} \prod_{t=1}^{n} Pr(\epsilon_t)$$

$$Pr(\epsilon_t) = \Phi( \epsilon_t )$$





\end{document}