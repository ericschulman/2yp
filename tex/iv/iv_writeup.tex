\documentclass{article}

%math stuff
\usepackage{amsmath}
\usepackage{enumitem}
\usepackage{mathtools}
\usepackage{listings}

%bibliography/appendix
\usepackage{cite}
\usepackage[toc,page]{appendix}

%figures
\usepackage{graphicx}
\usepackage{booktabs}

%General Formating
\renewcommand*\familydefault{\sfdefault}
\usepackage{cmbright}
\usepackage[letterpaper, portrait, margin=1.5in]{geometry}
\usepackage{fancyhdr}
\pagestyle{fancy}

%Header
\lhead{Schulman}
\rhead{Page \thepage}

\title{Predicting Volume Changes Using Price Data}
\author{Eric Schulman}
\date{\today}

\begin{document}




\maketitle

\section{Introduction}

Estimating the effect of price changes on volume is a difficult. Consumers adjust their spending practices to prices changes.  However, firms also adjust their pricing based on consumers spending habits. Firms compensate when volume is lower than average levels by cutting prices. More over, firms raise prices when consumers when volume is high bringing price back to average levels.

The literature on consumption suggests consumers smooth their consumptions. In other words, consumers want roughly the same amount every period; you would expect volume to be constant from week to week -- even if prices are changing. In fact, firms can coordinate their pricing activity around consumption smoothing.

If price changes where truly random, you could estimate how quantity responds to price changes. In lieu of that, you can estimate how quantity responds to prices by isolating 'random' price changes from 'non-random' ones. You can isolate 'random' price changes by comparing prices at retailers with the manufacturers' planned schedule. Then you can use the 'random' aspect of price changes to explain changes in quantity. This approach is called 'instrumental variables' or 'structural equations'.

\section{Model}

Formally, you can isolate the random aspect of price changes using the regression model.

$$\Delta p^{\text{retailer}}_{it} = \alpha_0 + \alpha_1 \Delta p^{\text{manufactuer}}_{it} + v_i$$

By fitting the model, you decompose $\Delta p^{\text{retailer}}_{it}$ into two parts. One is random, the other is a function of $\Delta p^{\text{manufactuer}}_{it}$.

$$\Delta  p^{\text{retailer}}_{it} = (\hat{\alpha_0} + \hat{\alpha_1} \Delta p^{\text{manufactuer}}_{it}) + (\hat{v_{it}}) $$

Isolating the 'random' part gives us:

$$\hat{v_{it}} = \Delta  p^{\text{retailer}}_{it}  - (\hat{\alpha_0} + \hat{\alpha_1} \Delta p^{\text{manufactuer}}_{it})  $$

We use the 'random' part of $p^{\text{retailer}}_{it}$ to estimate quantity changes with respect to price.

$$Q_{it} = \beta_0 + \beta_1 \hat{v_{it}} + \epsilon$$


\end{document}